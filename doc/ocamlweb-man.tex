\documentclass[12pt]{article}
\usepackage{fullpage}

\newcommand{\Caml}{\textsf{Caml}}
\newcommand{\ocamlweb}{\textsf{ocamlweb}}

\begin{document}

%%% titre %%%%%%%%%%%%%%%%%%%%%%%%%%%%%%%%%%%%%%%%%%%%%%%%%%
\title{ocamlweb: a literate programming tool \\ 
       for Objective Caml}
\author{Jean-Christophe Filli\^{a}tre \\
        \normalsize\texttt{www.lri.fr/\~{}filliatr}}
\date{}
\maketitle
%%%%%%%%%%%%%%%%%%%%%%%%%%%%%%%%%%%%%%%%%%%%%%%%%%%%%%%%%%%%

\section{Introduction}

%%%%%%%%%%%%%%%%%%%%%%%%%%%%%%%%%%%%%%%%%%%%%%%%%%%%%%%%%%%%

\section{Principles}

Documentation is inserted into \Caml\ files as \emph{comments}.


%%%%%%%%%%%%%%%%%%%%%%%%%%%%%%%%%%%%%%%%%%%%%%%%%%%%%%%%%%%%

\section{Usage}

\ocamlweb\ is invoked with a shell command line as
\begin{displaymath}
  \texttt{ocamlweb }<\textit{options and files}>
\end{displaymath}
Any command line argument which is not an option is considered as a
file. A file that is not a \Caml\ file is considered as a \LaTeX\ file,
and will be copied `as is' in the final document. The order of files
on the command line is kept in the final document. 

\subsection*{Command line options}

\begin{description}

\item[\texttt{-header}:]

  Do not skip the header of \Caml\ files. The default behavior is to
  skip them, since there are usually made of copyright and license
  informations, which you do not want to see in the final document.
  Headers are identified as comments right at the beginning of the
  \Caml\ file, and are stopped by any character other then a space
  outside a comment or by an empty line. 

\item[\texttt{-o }\textit{file}:] 
  
  Redirect the output to file \textit{file}.

\item[\texttt{-nodoc}:]

  Suppress the header and trailer of the document. Thus, you can
  insert the resulting document into a larger document.

\item[\texttt{-h}:]

  Give a short summary of the options.

\item[\texttt{-v}:]

  Print the version and exit.

\end{description}

%%%%%%%%%%%%%%%%%%%%%%%%%%%%%%%%%%%%%%%%%%%%%%%%%%%%%%%%%%%%

\end{document}

%%% Local Variables: 
%%% mode: latex
%%% TeX-master: t
%%% End: 
